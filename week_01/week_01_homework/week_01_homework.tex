% Options for packages loaded elsewhere
\PassOptionsToPackage{unicode}{hyperref}
\PassOptionsToPackage{hyphens}{url}
%
\documentclass[
]{article}
\usepackage{amsmath,amssymb}
\usepackage{lmodern}
\usepackage{ifxetex,ifluatex}
\ifnum 0\ifxetex 1\fi\ifluatex 1\fi=0 % if pdftex
  \usepackage[T1]{fontenc}
  \usepackage[utf8]{inputenc}
  \usepackage{textcomp} % provide euro and other symbols
\else % if luatex or xetex
  \usepackage{unicode-math}
  \defaultfontfeatures{Scale=MatchLowercase}
  \defaultfontfeatures[\rmfamily]{Ligatures=TeX,Scale=1}
\fi
% Use upquote if available, for straight quotes in verbatim environments
\IfFileExists{upquote.sty}{\usepackage{upquote}}{}
\IfFileExists{microtype.sty}{% use microtype if available
  \usepackage[]{microtype}
  \UseMicrotypeSet[protrusion]{basicmath} % disable protrusion for tt fonts
}{}
\makeatletter
\@ifundefined{KOMAClassName}{% if non-KOMA class
  \IfFileExists{parskip.sty}{%
    \usepackage{parskip}
  }{% else
    \setlength{\parindent}{0pt}
    \setlength{\parskip}{6pt plus 2pt minus 1pt}}
}{% if KOMA class
  \KOMAoptions{parskip=half}}
\makeatother
\usepackage{xcolor}
\IfFileExists{xurl.sty}{\usepackage{xurl}}{} % add URL line breaks if available
\IfFileExists{bookmark.sty}{\usepackage{bookmark}}{\usepackage{hyperref}}
\hypersetup{
  pdftitle={week\_01\_homework},
  hidelinks,
  pdfcreator={LaTeX via pandoc}}
\urlstyle{same} % disable monospaced font for URLs
\usepackage[margin=1in]{geometry}
\usepackage{color}
\usepackage{fancyvrb}
\newcommand{\VerbBar}{|}
\newcommand{\VERB}{\Verb[commandchars=\\\{\}]}
\DefineVerbatimEnvironment{Highlighting}{Verbatim}{commandchars=\\\{\}}
% Add ',fontsize=\small' for more characters per line
\usepackage{framed}
\definecolor{shadecolor}{RGB}{248,248,248}
\newenvironment{Shaded}{\begin{snugshade}}{\end{snugshade}}
\newcommand{\AlertTok}[1]{\textcolor[rgb]{0.94,0.16,0.16}{#1}}
\newcommand{\AnnotationTok}[1]{\textcolor[rgb]{0.56,0.35,0.01}{\textbf{\textit{#1}}}}
\newcommand{\AttributeTok}[1]{\textcolor[rgb]{0.77,0.63,0.00}{#1}}
\newcommand{\BaseNTok}[1]{\textcolor[rgb]{0.00,0.00,0.81}{#1}}
\newcommand{\BuiltInTok}[1]{#1}
\newcommand{\CharTok}[1]{\textcolor[rgb]{0.31,0.60,0.02}{#1}}
\newcommand{\CommentTok}[1]{\textcolor[rgb]{0.56,0.35,0.01}{\textit{#1}}}
\newcommand{\CommentVarTok}[1]{\textcolor[rgb]{0.56,0.35,0.01}{\textbf{\textit{#1}}}}
\newcommand{\ConstantTok}[1]{\textcolor[rgb]{0.00,0.00,0.00}{#1}}
\newcommand{\ControlFlowTok}[1]{\textcolor[rgb]{0.13,0.29,0.53}{\textbf{#1}}}
\newcommand{\DataTypeTok}[1]{\textcolor[rgb]{0.13,0.29,0.53}{#1}}
\newcommand{\DecValTok}[1]{\textcolor[rgb]{0.00,0.00,0.81}{#1}}
\newcommand{\DocumentationTok}[1]{\textcolor[rgb]{0.56,0.35,0.01}{\textbf{\textit{#1}}}}
\newcommand{\ErrorTok}[1]{\textcolor[rgb]{0.64,0.00,0.00}{\textbf{#1}}}
\newcommand{\ExtensionTok}[1]{#1}
\newcommand{\FloatTok}[1]{\textcolor[rgb]{0.00,0.00,0.81}{#1}}
\newcommand{\FunctionTok}[1]{\textcolor[rgb]{0.00,0.00,0.00}{#1}}
\newcommand{\ImportTok}[1]{#1}
\newcommand{\InformationTok}[1]{\textcolor[rgb]{0.56,0.35,0.01}{\textbf{\textit{#1}}}}
\newcommand{\KeywordTok}[1]{\textcolor[rgb]{0.13,0.29,0.53}{\textbf{#1}}}
\newcommand{\NormalTok}[1]{#1}
\newcommand{\OperatorTok}[1]{\textcolor[rgb]{0.81,0.36,0.00}{\textbf{#1}}}
\newcommand{\OtherTok}[1]{\textcolor[rgb]{0.56,0.35,0.01}{#1}}
\newcommand{\PreprocessorTok}[1]{\textcolor[rgb]{0.56,0.35,0.01}{\textit{#1}}}
\newcommand{\RegionMarkerTok}[1]{#1}
\newcommand{\SpecialCharTok}[1]{\textcolor[rgb]{0.00,0.00,0.00}{#1}}
\newcommand{\SpecialStringTok}[1]{\textcolor[rgb]{0.31,0.60,0.02}{#1}}
\newcommand{\StringTok}[1]{\textcolor[rgb]{0.31,0.60,0.02}{#1}}
\newcommand{\VariableTok}[1]{\textcolor[rgb]{0.00,0.00,0.00}{#1}}
\newcommand{\VerbatimStringTok}[1]{\textcolor[rgb]{0.31,0.60,0.02}{#1}}
\newcommand{\WarningTok}[1]{\textcolor[rgb]{0.56,0.35,0.01}{\textbf{\textit{#1}}}}
\usepackage{graphicx}
\makeatletter
\def\maxwidth{\ifdim\Gin@nat@width>\linewidth\linewidth\else\Gin@nat@width\fi}
\def\maxheight{\ifdim\Gin@nat@height>\textheight\textheight\else\Gin@nat@height\fi}
\makeatother
% Scale images if necessary, so that they will not overflow the page
% margins by default, and it is still possible to overwrite the defaults
% using explicit options in \includegraphics[width, height, ...]{}
\setkeys{Gin}{width=\maxwidth,height=\maxheight,keepaspectratio}
% Set default figure placement to htbp
\makeatletter
\def\fps@figure{htbp}
\makeatother
\setlength{\emergencystretch}{3em} % prevent overfull lines
\providecommand{\tightlist}{%
  \setlength{\itemsep}{0pt}\setlength{\parskip}{0pt}}
\setcounter{secnumdepth}{-\maxdimen} % remove section numbering
\ifluatex
  \usepackage{selnolig}  % disable illegal ligatures
\fi

\title{week\_01\_homework}
\author{}
\date{\vspace{-2.5em}}

\begin{document}
\maketitle

\begin{enumerate}
\def\labelenumi{\arabic{enumi}.}
\tightlist
\item
  Load in Tidyverse \& Lubridate library
\end{enumerate}

\begin{Shaded}
\begin{Highlighting}[]
\FunctionTok{library}\NormalTok{(tidyverse)}
\end{Highlighting}
\end{Shaded}

\begin{verbatim}
## Warning in as.POSIXlt.POSIXct(Sys.time()): unknown timezone '%Y %m %d'
\end{verbatim}

\begin{verbatim}
## -- Attaching packages --------------------------------------- tidyverse 1.3.1 --
\end{verbatim}

\begin{verbatim}
## v ggplot2 3.3.5     v purrr   0.3.4
## v tibble  3.1.2     v dplyr   1.0.7
## v tidyr   1.1.3     v stringr 1.4.0
## v readr   1.4.0     v forcats 0.5.1
\end{verbatim}

\begin{verbatim}
## -- Conflicts ------------------------------------------ tidyverse_conflicts() --
## x dplyr::filter() masks stats::filter()
## x dplyr::lag()    masks stats::lag()
\end{verbatim}

\begin{Shaded}
\begin{Highlighting}[]
\FunctionTok{library}\NormalTok{(lubridate)}
\end{Highlighting}
\end{Shaded}

\begin{verbatim}
## 
## Attaching package: 'lubridate'
\end{verbatim}

\begin{verbatim}
## The following objects are masked from 'package:base':
## 
##     date, intersect, setdiff, union
\end{verbatim}

\begin{enumerate}
\def\labelenumi{\arabic{enumi}.}
\setcounter{enumi}{1}
\tightlist
\item
  Assign books.csv to books\_df
\end{enumerate}

\begin{Shaded}
\begin{Highlighting}[]
\NormalTok{books\_df }\OtherTok{\textless{}{-}} \FunctionTok{read\_csv}\NormalTok{(}\StringTok{"data/books.csv"}\NormalTok{)}
\end{Highlighting}
\end{Shaded}

\begin{verbatim}
## 
## -- Column specification --------------------------------------------------------
## cols(
##   rowid = col_double(),
##   bookID = col_double(),
##   title = col_character(),
##   authors = col_character(),
##   average_rating = col_double(),
##   isbn = col_character(),
##   isbn13 = col_character(),
##   language_code = col_character(),
##   num_pages = col_double(),
##   ratings_count = col_double(),
##   text_reviews_count = col_double(),
##   publication_date = col_character(),
##   publisher = col_character()
## )
\end{verbatim}

\begin{enumerate}
\def\labelenumi{\arabic{enumi}.}
\setcounter{enumi}{2}
\tightlist
\item
  Perform some initial checks and view dataframe
\end{enumerate}

\begin{Shaded}
\begin{Highlighting}[]
\FunctionTok{str}\NormalTok{(books\_df)}
\end{Highlighting}
\end{Shaded}

\begin{verbatim}
## spec_tbl_df [11,123 x 13] (S3: spec_tbl_df/tbl_df/tbl/data.frame)
##  $ rowid             : num [1:11123] 1 2 3 4 5 6 7 8 9 10 ...
##  $ bookID            : num [1:11123] 1 2 4 5 8 9 10 12 13 14 ...
##  $ title             : chr [1:11123] "Harry Potter and the Half-Blood Prince (Harry Potter  #6)" "Harry Potter and the Order of the Phoenix (Harry Potter  #5)" "Harry Potter and the Chamber of Secrets (Harry Potter  #2)" "Harry Potter and the Prisoner of Azkaban (Harry Potter  #3)" ...
##  $ authors           : chr [1:11123] "J.K. Rowling/Mary GrandPré" "J.K. Rowling/Mary GrandPré" "J.K. Rowling" "J.K. Rowling/Mary GrandPré" ...
##  $ average_rating    : num [1:11123] 4.57 4.49 4.42 4.56 4.78 3.74 4.73 4.38 4.38 4.22 ...
##  $ isbn              : chr [1:11123] "0439785960" "0439358078" "0439554896" "043965548X" ...
##  $ isbn13            : chr [1:11123] "9780439785969" "9780439358071" "9780439554893" "9780439655484" ...
##  $ language_code     : chr [1:11123] "eng" "eng" "eng" "eng" ...
##  $ num_pages         : num [1:11123] 652 870 352 435 2690 ...
##  $ ratings_count     : num [1:11123] 2095690 2153167 6333 2339585 41428 ...
##  $ text_reviews_count: num [1:11123] 27591 29221 244 36325 164 ...
##  $ publication_date  : chr [1:11123] "9/16/2006" "9/1/2004" "11/1/2003" "5/1/2004" ...
##  $ publisher         : chr [1:11123] "Scholastic Inc." "Scholastic Inc." "Scholastic" "Scholastic Inc." ...
##  - attr(*, "spec")=
##   .. cols(
##   ..   rowid = col_double(),
##   ..   bookID = col_double(),
##   ..   title = col_character(),
##   ..   authors = col_character(),
##   ..   average_rating = col_double(),
##   ..   isbn = col_character(),
##   ..   isbn13 = col_character(),
##   ..   language_code = col_character(),
##   ..   num_pages = col_double(),
##   ..   ratings_count = col_double(),
##   ..   text_reviews_count = col_double(),
##   ..   publication_date = col_character(),
##   ..   publisher = col_character()
##   .. )
\end{verbatim}

\begin{Shaded}
\begin{Highlighting}[]
\NormalTok{books\_df }\SpecialCharTok{\%\textgreater{}\%} 
  \FunctionTok{summarise}\NormalTok{(}
    \AttributeTok{na\_count =} \FunctionTok{sum}\NormalTok{(}\FunctionTok{is.na}\NormalTok{(.))}
\NormalTok{    )}
\end{Highlighting}
\end{Shaded}

\begin{verbatim}
## # A tibble: 1 x 1
##   na_count
##      <int>
## 1        0
\end{verbatim}

\begin{enumerate}
\def\labelenumi{\arabic{enumi}.}
\setcounter{enumi}{3}
\tightlist
\item
  Rename selected columns in accordance with style guide \& tidy up
  language code format
\end{enumerate}

\begin{Shaded}
\begin{Highlighting}[]
\NormalTok{books\_df\_rows }\OtherTok{\textless{}{-}}\NormalTok{ books\_df }\SpecialCharTok{\%\textgreater{}\%} 
  \FunctionTok{rename}\NormalTok{(}\StringTok{"row\_id"} \OtherTok{=} \StringTok{"rowid"}\NormalTok{,}
         \StringTok{"book\_id"} \OtherTok{=} \StringTok{"bookID"}\NormalTok{,}
         \StringTok{"isbn\_13"} \OtherTok{=} \StringTok{"isbn13"}
\NormalTok{         ) }\SpecialCharTok{\%\textgreater{}\%} 
  \FunctionTok{mutate}\NormalTok{(}\AttributeTok{language\_code =} \FunctionTok{recode}\NormalTok{(language\_code,}
                                \StringTok{"eng"} \OtherTok{=} \StringTok{"en"}\NormalTok{,}
                                \StringTok{"en{-}US"} \OtherTok{=} \StringTok{"en{-}us"}\NormalTok{,}
                                \StringTok{"en{-}GB"} \OtherTok{=} \StringTok{"en{-}gb"}\NormalTok{,}
                                \StringTok{"en{-}CA"} \OtherTok{=} \StringTok{"en{-}ca"}\NormalTok{)}
\NormalTok{)}
\end{Highlighting}
\end{Shaded}


\end{document}
